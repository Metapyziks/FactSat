\documentclass[a4paper,12pt]{article}

\usepackage{lipsum}
\usepackage[english]{babel}
\usepackage[margin=24mm]{geometry}
\usepackage{amsmath}
\usepackage{amssymb}
\usepackage{hyperref}

\begin{document}
\begin{center}
\begin{minipage}{0.48\textwidth}
\begin{flushleft}
\small{James King}
\vspace{5mm}
\end{flushleft}
\end{minipage}
\begin{minipage}{0.48\textwidth}
\begin{flushright}
\small{\today}
\vspace{5mm}
\end{flushright}
\end{minipage}
\large{\textbf{Integer Factorization / SAT Solving Coursework}}
\end{center}

\section*{Factors Found}
\begin{tabular}{r r rcl c l}
\# & Product & Result & & & Solver & Time \\
1 & 14,843,833 & \textbf{UNSAT} & & & minisat & 0.343s \\
2 & 549,221,821 & 14,843,833 &x& 37 & minisat & 1.734s \\
3 & 961,748,941 & \textbf{UNSAT} & & & minisat & 6.328s \\
4 & 57,396,757,499 & 271,109 &x& 211,711 & minisat & 69.125s \\
5 & 1,047,090,939,649 & 1,079,621 &x& 969,869 & minisat & 163.218s \\
6 & 10,685,266,071,481 & 15,467,041 &x& 690,841 & minisat & 160.015s \\
7-12 & & & & & minisat & TIMEOUT ($>$250s) \\
\end{tabular}

\section*{SAT Solver Implementation}
I had a quick try at implementing a solver, although didn't have much time to optimise it. It's a recursive DPLL solver using MOMS as a heuristic for splitting, implemented in C$^\sharp$.

\vspace{5mm}
\url{https://github.com/Metapyziks/FactSat}

\vspace{5mm}
\noindent
Because the current approach is pretty inefficient (although gives correct solutions) I'll probably attempt a rewrite when I have the time.
\end{document}